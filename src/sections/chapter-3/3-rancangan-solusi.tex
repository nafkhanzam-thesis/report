\section{Rancangan Solusi}

% General Dataset Construction
Pada masalah kurangnya jumlah dataset yang digunakan sebagai data latih model, dibangun kumpulan dataset silver untuk pelatihan model \amrparsing{} lintas bahasa.
Diagram alur keseluruhan proses konstruksi dataset gold dan silver dapat dilihat pada \cref{fig:3-dataset-construction}.
Alur proses konstruksi dataset berkualitas silver mengikuti framework augmentasi data oleh \textcite{lee2022}.
Bagian 80\% dari dataset silver akan digunakan sebagai data latih dan bagian 20\% akan digunakan sebagai data validasi.

\figpdf[1][4]{3-dataset-construction}
  {drawio.pdf}
  {Diagram alur untuk konstruksi dataset gold dan silver dari korpus paralel dan dataset AMR.}

% Dataset Silver Parallel Corpora
Korpus paralel PANL-BPPT dan IWSLT2017 digunakan sebagai dataset pasangan kalimat Bahasa Inggris dan Bahasa Indonesia.
Kalimat berbahasa Inggris dari korpus paralel diubah menjadi graf \AMR{}.
Model \amrparsing{} yang digunakan adalah teknik \ti{ensembling} dari 5 model \sota{} \citek{lee2022}.
Hasil graf \AMR{} yang dihasilkan akan digunakan sebagai dataset latih berkualitas silver.
Diagram alur untuk konstruksi dataset par silver dari korpus paralel dapat dilihat pada \cref{fig:3-dataset-par-silver-construction}.

\figpdf[0.5][7]{3-dataset-par-silver-construction}
  {drawio.pdf}
  {Diagram alur untuk konstruksi dataset par silver dari korpus paralel.}

% Dataset Silver AMR 3.0
Dataset \AMR{} berkualitas gold yang digunakan adalah Dataset AMR 3.0 (LDC2020T02).
Augmentasi data untuk konstruksi dataset berkualitas silver mengambil bagian Train dan Dev \ti{split} dari AMR 3.0.
Kalimat berbahasa Inggris dari dataset \AMR{} tersebut diubah menjadi kalimat berbahasa Indonesia dengan menggunakan mesin translasi.
Model mesin translasi yang digunakan adalah model {OPUS-MT} \citek{tiedemann2020}.
Graf \AMR{} berkualitas gold juga dilakukan \ti{parsing} menjadi kalimat berbahasa Inggris (AMR-to-text) untuk menambah jumlah variasi pasangan dataset.
Model AMR-to-text yang digunakan adalah \gls{AMRBART}.
Kalimat berbahasa Inggris hasil \ti{parsing} tersebut juga diubah menjadi kalimat berbahasa Indonesia dengan menggunakan mesin translasi.
Maka akan terdapat dua pasangan kalimat berbahasa Inggris dan Indonesia untuk satu buah graf \AMR{}.
Dua pasangan tersebut dilakukan \ti{cross join} untuk menghasilkan empat pasang data latih berkualitas silver.

% Quality measurement and filtration.
Dataset trans silver tersebut kemudian dilakukan filtrasi dengan algoritma 1-NN dengan \cossim{} sebagai fungsi jarak antar representasi kalimatnya.
Dataset tersebut juga dihitung kualitasnya menggunakan rata-rata \cossim{}.
Diagram alur untuk konstruksi dataset trans silver dari AMR 3.0 dapat dilihat pada \cref{fig:3-dataset-trans-silver-construction}.

\figpdf[0.5][6]{3-dataset-trans-silver-construction}
  {drawio.pdf}
  {Diagram alur untuk konstruksi dataset trans silver dari AMR 3.0.}

% Dataset Gold
Augmentasi data untuk konstruksi dataset berkualitas gold mengambil bagian Test \ti{split} dari AMR 2.0.
Dataset AMR 2.0 digunakan supaya dapat hasil kinerja model dapat dikomparasikan dengan penelitian \textcite{putra2022}.
Bagian kalimat berbahasa Inggris dari Test \ti{split} ditranslasi menjadi Bahasa Indonesia oleh ahli.
Hasil kalimat berbahasa Indonesia tersebut berkualitas gold dan digunakan sebagai dataset test.
Diagram alur untuk konstruksi dataset gold dari AMR 2.0 dapat dilihat pada \cref{fig:3-dataset-gold-construction}.

\figpdf[0.5][5]{3-dataset-gold-construction}
  {drawio.pdf}
  {Diagram alur untuk konstruksi dataset gold dari AMR 2.0.}

\newcommand\graphMasked{\codesm{\textcolor[rgb]{0,.33,0}{<g>$g_1$,..[mask]..,$g_m$</g>}}}
\newcommand\graphBarMasked{\codesm{\textcolor[rgb]{0,.33,0}{<g>[mask]</g>}}}
\newcommand\idTagged[1]{\codesm{\textcolor{purple}{<id>#1</id>}}}
\newcommand\enTagged[1]{\codesm{\textcolor{blue}{<en>#1</en>}}}
\newcommand\sTagged[1]{\codesm{\textcolor{purple}{<s>#1</s>}}}

% Training Techniques
Skema pelatihan yang digunakan hanya untuk yang melakukan \amrparsing{} dengan menggunakan skema \ti{bilingual}.
Ilustrasi pelatihan dengan skema \ti{bilingual} dapat dilihat pada \cref{fig:3-bilingual-training-schema}
Pelatihan dicoba dengan konfigurasi dengan dan tanpa \ti{concatenating}.
Pada konfigurasi tanpa \ti{concatenating} tetap mengikuti teknik pelatihan sebelumnya yang dapat dilihat pada \cref{tab:3-pt-ft-strategies}.
\code{$t/g$} merupakan teks/graf \ti{original}.
\code{$\hat{t}/\hat{g}$} merupakan teks/graf yang \ti{noisy} (hasil \denoising{}).
\code{$\bar{t}/\bar{g}$} merupakan teks/graf yang hilang.
Semua output dari pelatihan ini adalah bentuk graf yang utuh.
Sedangkan pada konfigurasi dengan \ti{concatenating}, diperlukan sedikit modifikasi untuk melakukan \pretraining{} graf pada teknik \gls{AMRBART}.
Token \code{<s>} yang digunakan pada \gls{AMRBART} diganti menjadi kode bahasa yang digunakan.
Bahasa Indonesia ditandai dengan token \code{<id>} dan Bahasa Inggris ditandai dengan token \code{<en>}.
Kalimat berbahasa Indonesia dan Inggris dilinearisasi secara sejajar sebagai input \pretraining{} maupun \finetuning{}.
Strategi \pretraining{} dan \finetuning{} untuk pelatihan graf model \amrparsing{} lintas bahasa tersebut dapat dilihat pada \cref{tab:3-pt-ft-strategies-with-concat}.
Output dari pelatihan tersebut merupakan graf \AMR{} yang dilinearisasi, yaitu \code{<g>$g_1$,..,$g_m$</g>}.
Linearisasi graf menggunakan algoritma \gls{DFS} \citek{bevilacqua2021}.

\figpdf[0.6][2]{3-bilingual-training-schema}
  {drawio.pdf}
  {Pelatihan dengan skema \ti{bilingual}.}

\begin{filecontents*}{3-pt-ft-strategies.csv}
Phase,Task,Input
PT,\code{$\bar{t}\hat{g}2g$},\makecell[cl]{\sTagged{[mask]}\graphMasked{}}
,\code{$t\hat{g}2g$},\makecell[cl]{\sTagged{$a_1$,..,$a_n$}\graphMasked{}}
,\code{$\hat{t}\hat{g}2g$},\makecell[cl]{\sTagged{$a_1$,..[mask]..,$a_n$}\graphMasked{}}
FT,\code{$t\bar{g}2g$},\makecell[cl]{\sTagged{$a_1$,..,$a_n$}\graphMasked{}}
\end{filecontents*}
\tabl{3-pt-ft-strategies}
  {3-pt-ft-strategies.csv}
  {Strategi \pretraining{} (PT) dan \finetuning{} (FT) untuk pelatihan graf model \amrparsing{} lintas bahasa pada skema pelatihan \ti{bilingual}.}

\begin{filecontents*}{3-pt-ft-strategies-with-concat.csv}
Phase,Task,Input
PT,\code{$\bar{t}_{_{ID}}\bar{t}{_{_{EN}}}\hat{g}2g$},\makecell[cl]{\idTagged{[mask]}\enTagged{[mask]}\graphMasked{}}
,\code{$t_{_{ID}}\bar{t}{_{_{EN}}}\hat{g}2g$},\makecell[cl]{\idTagged{$a_1$,..,$a_{n_1}$}\enTagged{[mask]}\graphMasked{}}
,\code{$\bar{t}{_{_{ID}}}t_{_{EN}}\hat{g}2g$},\makecell[cl]{\idTagged{[mask]}\enTagged{$b_1$,..,$b_{n_2}$}\graphMasked{}}
,\code{$t_{_{ID}}t_{_{EN}}\hat{g}2g$},\makecell[cl]{\idTagged{$a_1$,..,$a_{n_1}$}\enTagged{$b_1$,..,$b_{n_2}$}\graphMasked{}}
,\code{$t_{_{ID}}\hat{t}{_{_{EN}}}\hat{g}2g$},\makecell[cl]{\idTagged{$a_1$,..,$a_{n_1}$}\enTagged{$b_1$,..[mask]..,$b_{n_2}$}\graphMasked{}}
,\code{$\hat{t}{_{_{ID}}}t_{_{EN}}\hat{g}2g$},\makecell[cl]{\idTagged{$a_1$,..[mask]..,$a_{n_1}$}\enTagged{$b_1$,..,$b_{n_2}$}\graphMasked{}}
,\code{$\hat{t}{_{_{ID}}}\hat{t}{_{_{EN}}}\hat{g}2g$},\makecell[cl]{\idTagged{$a_1$,..[mask]..,$a_{n_1}$}\enTagged{$b_1$,..[mask]..,$b_{n_2}$}\graphMasked{}}
FT,\code{$t_{_{ID}}\bar{t}{_{_{EN}}}\bar{g}{}2g$},\makecell[cl]{\idTagged{$a_1$,..,$a_{n_1}$}\enTagged{[mask]}\graphBarMasked{}}
,\code{$\bar{t}{_{_{ID}}}t_{_{EN}}\bar{g}{}2g$},\makecell[cl]{\idTagged{[mask]}\enTagged{$b_1$,..,$b_{n_2}$}\graphBarMasked{}}
\end{filecontents*}
\tabl{3-pt-ft-strategies-with-concat}
  {3-pt-ft-strategies-with-concat.csv}
  {Strategi \pretraining{} (PT) dan \finetuning{} (FT) untuk pelatihan graf model \amrparsing{} lintas bahasa pada skema pelatihan \ti{bilingual} dengan konfigurasi \ti{concatenating}.}

Pembangunan data latih silver ketika \pretraining{} dilakukan \ti{masking} secara dinamis ketika melakukan training \citek{bai2022}.
Fungsi \ti{noising} dilakukan secara acak untuk melakukan \ti{masking} sebuah subgraf, 15\% bagian untuk \ti{masking} simpul (konsep), dan/atau 15\% untuk \ti{masking} sisi (relasi).
Sebuah subgraf minimal memiliki satu simpul dan satu sisi.
Setiap jenis \ti{noising} tersebut memiliki kemungkinan yang bertambah setiap \ti{step} pelatihan.
Nilai kemungkinan tersebut dapat dilihat pada \cref{eq:masking-probability}.
Nilai $p$ merupakan kemungkinan dilakukan \ti{noising}.
Nilai $t$ merupakan iterasi pelatihan dan $T$ merupakan iterasi maksimal.

\eq{masking-probability}{p = 0.1 + 0.75 \times t/T}

Pelatihan \amrparsing{} akan dicoba pada beberapa \multil{} model.
Model-model yang akan dicoba adalah mBART, mT5, IndoBART, dan IndoT5.
Eksperimen \amrparsing{} juga akan dicoba dengan teknik \AMR{} ensembling \citek{hoang2021} dari hasil semua model-model tersebut.

% Evaluation
Evaluasi dilakukan secara kuantitatif dan kualitatif.
Metrik evaluasi yang digunakan untuk evaluasi kuantitatif adalah \SMATCH{}.
Evaluasi tersebut dilakukan dengan menghitung kemiripan graf \AMR{} hasil prediksi dan graf \AMR{} asli dari dataset test.
Metode kualitatif yang digunakan adalah evaluasi \transdiver{}.

% Baseline Evaluation
Pendekatan \ti{translate-and-parse} \citek{uhrig2021} digunakan sebagai baseline pada penelitian ini.
Model mesin translasi yang digunakan adalah {OPUS-MT} dan model \amrparsing{} yang digunakan adalah \gls{AMRBART}.
Tahapan yang dilakukan adalah melakukan translasi dari kalimat berbahasa Indonesia menjadi Bahasa Inggris, lalu dilakukan \amrparsing{} menjadi graf \AMR{}.
