\section{Analisis Masalah}

Model \gls{AMRBART} \citek{bai2022} merupakan model \sota{} yang dibangun untuk \amrparsing{} dan \AMRtoTEXT{} untuk Bahasa Inggris.
Belum dikembangkan sebuah model \amrparsing{} \crosslingual{} untuk Bahasa Indonesia berdasarkan model graf pralatih \gls{AMRBART}.

Pelatihan pembelajaran mesin untuk \ti{task} \amrparsing{} memerlukan dataset \ti{training} pasangan kalimat dan graf \AMR{}-nya.
Penelitian \textcite{putra2022} menggunakan dataset silver yang dihasilkan dari AMR 2.0 dan korpus paralel PANL-BPPT sebagai data latih \amrparsing{} lintas bahasa untuk Bahasa Indonesia.
Berdasarkan pada penelitian \textcite{lee2022}, model \amrparsing{} dapat meningkat kinerjanya dengan bertambahnya jumlah dataset silver.
Dataset yang digunakan \textcite{putra2022} secara kuantitas masih kurang karena ada korpus paralel IWSLT17 dan dataset AMR 3.0 yang belum digunakan.

Pembangunan dataset silver pada penelitian \textcite{putra2022} tidak dievaluasi kualitasnya.
Banyak potensi \ti{instance} data berkualitas buruk yang digunakan sebagai data latih, sehingga dapat mengurangi kinerja model.
Perhitungan kualitas tersebut juga dapat dimanfaatkan sebagai acuan dalam memperbaiki kualitas dataset silver untuk meningkatkan kinerja model \amrparsing{}.
