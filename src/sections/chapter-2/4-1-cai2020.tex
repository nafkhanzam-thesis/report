\subsection{AMR Parsing via Graph-sequence Iterative Inference \citek{cai2020}}

% \amrparsing{} merupakan suatu tugas mentransformasi teks natural language ke AMR.
Salah satu tantangan dalam \amrparsing{} adalah kurangnya pemetaan eksplisit antara simpul pada graf dan kata kata dalam teks.
Untuk saat ini, akurasi \ti{parsing} dari penelitian-penelitian terkait masih belum memuaskan dibandingkan kinerja manusia, terutama pada kasus dimana kalimat lebih panjang dan informatif.
Salah satu kemungkinan alasan dari kekurangan ini adalah kurangnya model interaksi antara prediksi konsep dan prediksi relasi yang penting untuk mendapatkan hasil yang tidak ambigu.

Pada tingkat dasar, kita dapat mengkategorikan pendekatan AMR parsing menjadi dua kelas, yaitu dua tahap \ti{parsing} dan satu tahap \ti{parsing} \citek{cai2020}.
Pada pendekatan dua tahap \ti{parsing} digunakan desain \ti{pipeline} untuk identifikasi keseluruhan konsep, lalu diikuti dengan prediksi relasi antar hasil prediksi konsep.
Pada pendekatan satu tahap \ti{parsing} dikategorikan menjadi tiga jenis, yaitu \ti{parsing} berbasis transisi, \ti{parsing} berbasis \gls{seq2seq}, dan \ti{parsing} berbasis graf.
Pendekatan satu tahap jenis \ti{parsing} berbasis transisi dilakukan dengan memproses kalimat dari kiri ke kanan dan membangun grafik secara bertahap dengan secara bergantian memasukkan simpul atau sisi baru.
Pendekatan satu tahap jenis \ti{parsing} berbasis \gls{seq2seq} dengan melihat \ti{parsing} sebagai transduksi urutan linear ke urutan linear juga dengan memanfaatkan linearisasi grafik AMR.
Pendekatan satu tahap jenis \ti{parsing} berbasis graf di mana setiap langkah waktu, simpul baru beserta koneksinya ke simpul yang ada diputuskan bersama secara berurutan maupun paralel.
\cref{fig:4-1-partially-constructing-amr-graph} merupakan contoh proses pembentukan graf \AMR{} dengan pendekatan satu tahap jenis \ti{parsing}.

\fig{4-1-partially-constructing-amr-graph}
  {sections/chapter-2/4-1-partially-constructing-amr-graph.png}
  {
    Pembangunan sebuah graf \AMR{} dari subgraf \AMR{} yang sebagian terbentuk \citek{cai2020}.
    Lanjutan kemungkinan tahap pembangunan dapat berupa ekspansi untuk: (a) konsep \quotei{boy} dengan relasi ARG0 atau (b) konsep negasi dengan relasi polarity.
  }

\textcite{cai2020} mengusulkan pendekatan \amrparsing{} via \ti{graph-sequence iterative inference} yang meniru proses manusia dalam melakukan deduksi graf semantik dari suatu kalimat.
Pendekatan ini dimulai dari graf kosong yang memanjang secara iteratif dari simpul ke simpul.
Ilustrasi pendekatan ini dapat dilihat pada \cref{fig:4-1-dual-graph-sequence-iterative-inference}.

% Sejauh ini, kausal timbal balik dan prediksi konsep masih belum dipelajari secara detail dan dimanfaatkan sepenuhnya.
% Pendekatan AMR Parsing via Graph-sequence Iterative Inference mengikuti proses ketika expert manusia melakukan deduksi graf semantik dari suatu kalimat.
% Output dari graf ini dimulai dari graf kosong yang memanjang secara inkremental dengan cara simpul ke simpul.
% Pendekatan AMR parsing ini merupakan sebuah seri berisi sekuens graf ganda dengan pendekatan inferensi iteratif untuk pembuatan keputusan dan desain.
% Ilustrasi sekuens graf ganda dengan inferensi iteratif ditampilkan pada \cref{fig:4-1-fig2}.

\fig[1]{4-1-dual-graph-sequence-iterative-inference}
  {sections/chapter-2/4-1-dual-graph-sequence-iterative-inference.png}
  {Ilustrasi dari pendekatan \ti{graph-sequence iterative inference} untuk \amrparsing{} \citek{cai2020}.}
