\subsection{\Glsfirst{AMRBART} \citek{bai2022}}

\Gls{PLM} telah terbukti dapat melakukan \ti{task} \amrparsing{} dan generasi AMR-to-text dengan baik.
Namun, \gls{PLM} umumnya dilatih pada data tekstual, sehingga tidak optimal untuk melakukan generasi data terstruktur seperti \AMR{}.
\textcite{bai2022} memperbaiki permasalahan tersebut dengan menambahkan strategi \pretraining{} pada model untuk mengintegrasikan informasi teks dan graf \AMR{}.
Model ini melinearisasi graf \AMR{} ke sekuens sehingga baik AMR parsing dan AMR-to-text generation dapat dilakukan menggunakan model \gls{seq2seq}.
Model ini melakukan pre-training pada struktur \AMR{} menggunakan \gls{BART}.

Tahapan \ti{preprocessing} dalam model ini mengadopsi penelitian \textcite{bevilacqua2021}, yaitu \ti{recategorization} untuk mengurangi ukuran \ti{vocab} untuk menangani \ti{sparsity} data.
\ti{Recategorization} dilakukan dengan menghapus simpul \ti{sense}, \ti{link} wiki, atribut polaritas, dan menganonimasi \ti{named entity}.
Walaupun \ti{recategorization} menunjukkan kinerja yang kurang pada penelitian \textcite{bevilacqua2021}, namun pada \gls{AMRBART} tidak terjadi pengurangan kinerja.
Lemmatisasi juga dilakukan pada fase \ti{preprocessing}.
Linearisasi graf menggunakan algoritma \gls{DFS} yang diadopsi dari penelitian \textcite{bevilacqua2021}.
Pembentukan graf \AMR{} hasil prediksi dilakukan pada tahapan \ti{postprocessing} dengan mengubah bentuk format \gls{DFS} menjadi bentuk format PENMAN.
Wikifikasi, pemberian \ti{link} pada atribut wiki, juga dilakukan menggunakan BLINK Entity Linker \citek{wu2020}.
Pengembalian atribut polaritas dilakukan dengan mengidentifikasi lemma yang bersifat negasi.

Model ini mengenalkan dua strategi \ti{self-supervised training} dalam melakukan \pretraining{} model \gls{BART} pada graf \AMR{}.
Dapat dilihat pada \cref{fig:4-3-pre-training-strategies}, strategi level \denoising{} simpul/sisi mendukung model untuk menangkap pengetahuan lokal mengenai simpul dan sisi.
Strategi \denoising{} level graf mengarahkan model untuk memprediksi sub-graf yang dapat memfasilitasi pembelajaran graf.
\begin{enumerate}
  \item \Denoising{} level simpul/sisi.
  Pengaplikasian fungsi noise pada simpul dan sisi AMR untuk mengkonstruksi input graf yang kotor.
  Fungsi noise ini diimplementasikan dengan \ti{masking} 15\% simpul dan 15\% sisi di setiap graf.

  \item \Denoising{} level sub-graf.
  \ti{Task} ini bertujuan untuk mengembalikan graf lengkap ketika diberikan sebagian dari graf.
  Metode ini menghilangkan sub-graf secara acak dari graf dan mengubahnya dengan token \ti{mask}.
\end{enumerate}

\fig[1]{4-3-pre-training-strategies}
  {sections/chapter-2/4-3-pre-training-strategies.png}
  {Ilustrasi strategi pre-training: (1) \denoising{} level simpul/sisi (a->b); (2) \denoising{} level sub-graf (c->b) \citek{bai2022}.}

Strategi \pretraining{} (P.T.) dan \finetuning{} (F.T.) dapat dilihat pada \cref{tab:4-3-pt-ft-strategies}.
\code{$t/g$} merupakan teks/graf \ti{original}.
\code{$\hat{t}/\hat{g}$} merupakan teks/graf yang \ti{noisy} (hasil \denoising{}).
\code{$\bar{t}/\bar{g}$} merupakan teks/graf yang hilang.
Token \code{<s>} dan \code{<g>} digunakan untuk membedakan informasi teks dan graf pada input.
\Pretraining{} dilakukan dengan menghapus sebagian dari input teks/graf dengan teknik \denoising{}.
\Finetuning{} dilakukan dengan menghapus keseluruhan dari teks/graf yang dituju.
Pada standard \pretraining{} dan \finetuning{}, input berupa salah satu dari teks atau graf dan outputnya berupa teks atau graf yang dituju.
Namun, pada unified \pretraining{} dan \finetuning{}, input berupa kedua teks dan graf, dan outputnya berupa teks atau graf yang dituju.

\tabl{4-3-pt-ft-strategies}
  {sections/chapter-2/4-3-pt-ft-strategies.csv}
  {Strategi \pretraining{} (P.T.) dan \finetuning{} (F.T.) untuk pelatihan graf \citek{bai2022}.}
