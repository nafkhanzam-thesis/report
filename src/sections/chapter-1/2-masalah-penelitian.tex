\section{Masalah Penelitian}

Teknik \crosslingual{} \amrparsing{} terbaru untuk Bahasa Indonesia adalah teknik oleh \textcite{putra2022} yang menggunakan teknik \gls{stog} oleh \citek{zhang2019}.
Banyak teknik-teknik lain yang memiliki kinerja \amrparsing{} yang lebih baik, seperti \gls{AMRBART} \citek{bai2022}.
Dataset yang digunakan oleh \textcite{putra2022} juga belum menggunakan dataset terbaru, yakni AMR 3.0, yang memiliki kalimat dan aturan anotasi lebih banyak dari AMR 2.0.
Terdapat korpus paralel IWSLT17 \citek{cettolo2017} yang belum digunakan oleh \textcite{putra2022}.

% Thesis Problem
Rumusan masalah dari tesis ini adalah bagaimana melakukan \amrparsing{} dari kalimat berbahasa Indonesia menjadi graf \AMR{} berbahasa Inggris dengan menggunakan dataset \AMR{} Bahasa Inggris, korpus paralel Bahasa Inggris-Indonesia, dan \multil{} \ti{language model} yang sudah ada.
