\clearpage
\phantomsection{}
\addcontentsline{toc}{chapter}{\uppercase{Abstrak}}

\begin{spacing}{1}
\begin{center}
  \textbf{\large \MakeUppercase{Abstrak}} \\[2em]

  \textbf{\large \MakeUppercase{\Title}} \\[2em]

  \normalsize \normalfont Oleh\\
  \large \bfseries \AuthorName\\
  NIM:~\uppercase{\AuthorNIM}\\
  (Program Studi Magister Informatika)\\[3em]
\end{center}

% persoalan
\Gls{AMR} merupakan salah satu representasi yang memperhatikan semantik dari sebuah teks.
\amrparsing{} untuk kalimat berbahasa Indonesia telah dikembangkan untuk \ti{parsing} berbasis aturan, \ti{dependency parser}, dan \crosslingual{}.
Anotasi AMR berbahasa Indonesia masih memiliki konsep dan relasi yang terbatas sehingga \AMR{} berbahasa Inggris digunakan.
Pendekatan \crosslingual{} mengubah kalimat berbahasa Indonesia menjadi \AMR{} berbahasa Inggris.
Penelitian sebelumnya menggunakan teknik pelatihan bilingual yaitu menggunakan dataset berbahasa Indonesia dan Inggris.
Penelitian sebelumnya menggunakan dataset AMR 2.0 dan korpus paralel PANL-BPPT dan belum dilakukan filtrasi data untuk meningkatkan kualitasnya.
Model yang digunakan pada penelitian sebelumnya bukan lagi merupakan model \sota{}.

% metodologi
Tahap awal dilakukan analisis permasalahan dalam melakukan \amrparsing{} dalam Bahasa Indonesia.
Hal ini dilakukan dengan mendalami materi mengenai teknik-teknik \amrparsing{} yang ada sebelumnya dan membandingkan hasil kinerjanya.
Kemudian dilakukan perancangan dan implementasi berdasarkan hasil analisis teknik-teknik \amrparsing{} sebelumnya.
Dirancang sebuah solusi dengan menggunakan beberapa gabungan teknik dan model tersebut.
Lalu dilakukan implementasi dari rancangan tersebut dengan melakukan training model terhadap dataset yang tersedia.
Hasil implementasi sebelumnya diuji dan dievaluasi secara kuantitatif dengan menggunakan metrik SMATCH untuk dibandingkan hasilnya dengan teknik sebelumnya.
Hasil \amrparsing{} juga dievaluasi secara kualitatif dengan analisis divergensi translasi.
Evaluasi secara kuantitatif dan kualitatif tersebut digunakan untuk kesimpulan kelayakan model \amrparsing{} cross-lingual untuk kalimat berbahasa Indonesia.

% solusi global
Pada penelitian ini dibangun model \amrparsing{} \crosslingual{} menggunakan teknik training AMRBART dengan model mBART dan mT5.
Dataset yang digunakan untuk \ti{training} adalah AMR 3.0 dan korpus paralel PANL-BPPT dan IWSLT17.
Teknik \ti{training} AMRBART juga dikombinasikan dengan teknik \ti{training} translation language model dengan tersedianya korpus paralel pasangan kalimat berbahasa Indonesia dan Inggris.

\noindent Kata kunci: \ti{abstract meaning representation}, \crosslingual{}, \ti{parsing}
\end{spacing}
