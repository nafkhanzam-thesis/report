\section{Pembangunan Dataset}

Pelatihan model \amrparsing{} membutuhkan dataset \ti{train}, \ti{dev}, dan \ti{test}.
Setiap data yang diperlukan adalah pasangan graf \AMR{}, kalimat berbahasa Inggris, dan kalimat berbahasa Indonesia.
Sumber dataset yang digunakan adalah AMR 3.0 train split, AMR 3.0 dev split, AMR 2.0 test split, Korpus paralel PANL-BPPT, dan Korpus paralel IWSLT17.
AMR 2.0 test split juga termasuk kalimat berbahasa Indonesia hasil translasi ahli yang didapat dari penelitian \textcite{putra2022}.
Jenis data dan tahapan pembangunan masing-masing dataset dapat dilihat pada \cref{tab:4.1.original-dataset}.

\begin{filecontents*}{4.1.original-dataset.csv}
  Split,Dataset,Jumlah,AMR,EN,ID
  \makecell[l]{Train},\makecell[l]{PANL-BPPT},\makecell[r]{24,024},,\checkmark{},\checkmark{}
  \makecell[l]{Train},\makecell[l]{IWSLT17},\makecell[r]{107,329},,\checkmark{},\checkmark{}
  \makecell[l]{Train},\makecell[l]{AMR 3.0 Train},\makecell[r]{55,635},\checkmark{},\checkmark{},
  \makecell[l]{Dev},\makecell[l]{AMR 3.0 Dev},\makecell[r]{1,722},\checkmark{},\checkmark{},
  \makecell[l]{Test},\makecell[l]{AMR 2.0 Test},\makecell[r]{1,371},\checkmark{},\checkmark{},\checkmark{}
\end{filecontents*}
\tabl{4.1.original-dataset}
  {4.1.original-dataset.csv}
  {Detail ketersediaan data dari masing-masing dataset.}

Tahapan yang dilakukan dalam membangun pasangan graf \AMR{}, kalimat berbahasa Inggris, dan kalimat berbahasa Indonesia dari kumpulan dataset tersebut adalah:

\begin{enumerate}
  \item Melakukan \amrparsing{} kalimat berbahasa Inggris dari Korpus paralel PANL-BPPT dan Korpus paralel IWSLT17 menggunakan algoritma MBSE \citek{lee2022}.
  Proses ini menghasilkan \AMR{} dengan format PENMAN (\code{AMR}).
  \item Melinearisasi semua graf \AMR{} dengan format DFS (\code{AMR-DFS}).
  \item Melakukan \AMRtoTEXT{} graf \AMR{} dari dataset AMR 3.0 Train dan AMR 3.0 Dev untuk mendapatkan alternatif kalimat berbahasa Inggris (\code{EN-ALT}).
  \item Melakukan translasi semua kalimat berbahasa Inggris original dan alternatif dari dataset AMR 3.0 Train dan AMR 3.0 Dev menjadi kalimat berbahasa Indonesia (\code{ID} dan \code{ID-ALT}).
  \item Melakukan translasi kembali semua kalimat berbahasa Indonesia dari tahap 4 menjadi kalimat berbahasa Inggris (\code{EN-BACK} dan \code{EN-ALT-BACK}).
\end{enumerate}

Tahapan pembangunan kalimat alternatif menggunakan \AMRtoTEXT{} menambah jumlah AMR 3.0 Train dan AMR 3.0 Dev menjadi dua kali lipat, yaitu 111,270 dan 3,444.
Hasil dataset yang dibangun kemudian dievaluasi dan difiltrasi dari yang buruk untuk mendapatkan kualitas dataset yang baik sebagai data latih.
Dataset yang dilakukan evaluasi dan filtrasi adalah AMR 3.0 Train dan AMR 3.0 Dev.
Evaluasi kualitas dataset dilakukan dengan menghitung tiga jenis skor evaluasi sebagai berikut:

\begin{enumerate}
  \item \Cossim{} dari \ti{sentence embedding} kalimat berbahasa Indonesia dan kalimat berbahasa Inggris menggunakan \gls{LABSE}: \code{EN} dengan \code{ID} dan \code{EN-ALT} dengan \code{ID-ALT}.
  \item Peringkat \ti{nearest neighbor} kedekatan \ti{sentence embedding} dari kalimat berbahasa Indonesia dengan pasangan kalimat berbahasa Inggris dibandingkan dengan kalimat berbahasa Inggris lainnya: \code{ID} dengan \code{EN} dan \code{ID-ALT} dengan \code{EN-ALT}.
  \item Skor \BLEU{} dari pasangan kalimat berbahasa Inggris dan kalimat berbahasa Inggris hasil translasi kembali dari Bahasa Indonesia: \code{EN} dengan \code{EN-BACK} dan \code{EN-ALT} dengan \code{EN-ALT-BACK}.
  \item Skor \BLEU{} dari pasangan kalimat berbahasa Inggris dan kalimat alternatif berbahasa Inggris dari hasil \AMRtoTEXT{}: \code{EN} dengan \code{EN-ALT}.
\end{enumerate}

Berdasarkan hasil evaluasi tersebut dilakukan filtrasi data untuk menghasilkan kualitas data latih yang lebih baik.
Filtrasi-filtrasi data yang dilakukan adalah sebagai berikut:

\begin{enumerate}
  \item Skor \BLEU{} antara \code{EN} dan \code{EN-ALT} di antara 0.1 dan 0.9 dengan tujuan menghindari kalimat alternatif yang terlalu mirip dan yang terlalu berbeda \citek{lee2022}.
  \item Peringkat \ti{nearest neighbor} kedekatan \ti{sentence embedding} dari kalimat berbahasa Indonesia dengan pasangan kalimat berbahasa Inggris dibandingkan dengan kalimat berbahasa Inggris lainnya yang memiliki kedekatan peringkat satu (\gls{1-NN}) \citek{blloshmi2020}.
\end{enumerate}

Hasil evaluasi sebelum dilakukan filtrasi dapat dilihat pada \cref{tab:4.1.before-filtration}.
Hasil evaluasi setelah dilakukan filtrasi dapat dilihat pada \cref{tab:4.1.after-filtration}.
Jumlah total data untuk masing-masing split (train, dev, dan test) yang digunakan sebagai data pelatihan model dapat dilihat pada \cref{tab:4.1.total-dataset-count}.

\begin{filecontents*}{4.1.before-filtration.csv}
  Dataset,Jumlah,Cosine Similarity,BLEU-back,BLEU-alt
  \makecell[l]{AMR 3.0 Train},\makecell[r]{111,270},0,0,0
  \makecell[l]{AMR 3.0 Dev},\makecell[r]{3,444},0,0,0
\end{filecontents*}
\tabl{4.1.before-filtration}
  {4.1.before-filtration.csv}
  {Hasil evaluasi dataset sebelum dilakukan filtrasi.}

\begin{filecontents*}{4.1.after-filtration.csv}
  Dataset,Jumlah,Cosine Similarity,BLEU-back,BLEU-alt
  \makecell[l]{AMR 3.0 Train},\makecell[r]{55,635},0,0,0
  \makecell[l]{AMR 3.0 Dev},\makecell[r]{1,722},0,0,0
\end{filecontents*}
\tabl{4.1.after-filtration}
  {4.1.after-filtration.csv}
  {Hasil evaluasi dataset sebelum dilakukan filtrasi.}

\begin{filecontents*}{4.1.total-dataset-count.csv}
  Split,Dataset,Jumlah,Total
  \makecell[l]{Train},\makecell[l]{PANL-BPPT},\makecell[r]{24,024},\makecell[r]{0}
  ,\makecell[l]{IWSLT17},\makecell[r]{107,329},
  ,\makecell[l]{AMR 3.0 Train},\makecell[r]{0},
  \makecell[l]{Dev},\makecell[l]{AMR 3.0 Dev},\makecell[r]{0},\makecell[r]{0}
  \makecell[l]{Test},\makecell[l]{AMR 2.0 Test},\makecell[r]{1,371},\makecell[r]{1,371}
\end{filecontents*}
\tabl{4.1.total-dataset-count}
  {4.1.total-dataset-count.csv}
  {Jumlah total masing-masing split dataset.}
